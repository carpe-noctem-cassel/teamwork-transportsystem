\documentclass[a4paper,12pt,headsepline,toc=flat]{scrartcl}
\usepackage{amssymb}
\setcounter{tocdepth}{3}
\usepackage{graphicx}
\usepackage{rotating}
\usepackage{float}
\usepackage{url}
\usepackage{hyperref}
\usepackage{array}
\usepackage{longtable}
\usepackage{listings}
\usepackage{color}
\usepackage[utf8]{inputenc}
\usepackage{setspace}
\usepackage[ngerman]{babel}

\begin{document}
	
	\begin{verbatim}
	
	
	\end{verbatim}
	
	\begin{center}
		\Large{Universität Kassel}\\
		\Large{Fachbereich 16 - Informatik und Elektrotechnik}\\
	\end{center}
	
	
	\begin{verbatim}
	
	
	
	
	\end{verbatim}
	\begin{center}
		\doublespacing
		\textbf{\LARGE{Teamarbeit}}\\
		\singlespacing
		\begin{verbatim}
		
		\end{verbatim}
		\textbf{Abschlussbericht}
	\end{center}
	\begin{verbatim}
	
	\end{verbatim}
	\begin{center}
		
	\end{center}
	\begin{verbatim}
	
	\end{verbatim}
	\begin{center}
		
	\end{center}
	\begin{verbatim}
	
	
	
	
	\end{verbatim}
	\begin{flushleft}
		\begin{tabular}{llll}
			\textbf{Autoren:} & & Dennis & \\
			& & Robert Meschkat & \\
			& & Philipp Schenk & \\
			& & Eric Wagner & \\ \\
			\textbf{Betreuer:} & & M. Sc. Stephan Opfer &\\
		\end{tabular}
	\end{flushleft}
	\newpage
	
	\tableofcontents
	\newpage
	
	\section{Einleitung}
		Philipp\\
		REMOVE THIS: \href{https://docs.google.com/document/d/1wGlFley6lwhnpLsj8ms7M6fnOjplazlHNr8Bfbv3vso/edit}{Beispiel}\\\\
		Kurze Beschreibung des Themas Teamarbeit. Was ist der Bericht?
	\newpage
	\section{Technische Arbeit}
	
	\subsection{Arbeitsauftrag}
		Robert
		\begin{itemize}
			\item Erstellen eines Transporters aus einem Turtlebot
			\item Anbindung eines Drucksensors unter einen Tragekorb
			\item Integration der Sensorwerte in das bestehende Framework
			\item UI zum Steuern entwickeln
			\item Testen und Evaluieren
		\end{itemize}
	
	\subsection{Programmentwurf}
		Robert\\
		\begin{itemize}
			\item Turtlebot beschreiben
			\begin{itemize}
				\item Was ist der Turtlebot?
				\item Was kann er?
				\item Was wurde im Fachgebiet schon damit gemacht?
			\end{itemize}
			
			\item Verwendete Soft- und Hardware
			\begin{itemize}
				\item Verwendung von Linux 16.04 als Betriebssystem
				\item Roboterprogrammierung in C++
				\item Verschiedene existierende Repositories und GIT
				\item Verwenden von QT oder Chromium für das Interface
				\item Verwenden von Arduino mit Taster als Sensor
				\item Rosserial Arduino zur Umwandlung in Nachrichten
				\item Haribo-Korb zum Tragen
				\item Steuerung des Roboters über Nachrichten aus der UI
				\item Optionale Idee: RFID-Leser mit Tags in Tassen oder Büchern
			\end{itemize}
			
			\item Vorwissen der Teammitglieder beschreiben (hier oder bei Teamrollen?)
			\begin{itemize}
				\item Robert: Erfahrungen in Linux
				\item Eric: Erfahrungen in UI-Programmierung
				\item Dennis: Erfahrungen mit Tastern und Hardware
			\end{itemize}
		\end{itemize}
	\subsection{Umsetzung}
		Eric (Bis UI)\\
		\begin{itemize}
			\item Am Anfang Beschäftigung mit den Themen und Einarbeitung (Installation von Ubuntu und ROS)
			\item Besprechung mit dem Betreuer zu Konkretisierung des Auftrags
			\item Grafik vom 14.05. einbauen und beschreiben 
			\item Entscheidung für QT und RQT für die UI (Beschreiben)
			\item Probleme durch Betriebssystem und Branches erwähnen
			\item Parallele Arbeit an Taster und UI
			\item Nachdem Laptop nicht funktioniert hat wurde auf den Rechner umgeschwenkt
			\item (Vielleicht UI und Taster in zwei Unterkapitel teilen)
		\end{itemize}
			\subsubsection{UI}
			\begin{itemize}
				\item Verwenden von QTCreator zum Erstellen des UI-Fensters
				\item Schreiben des Codes mit C++ und ROS (RQT)
				\item Erstellen der UI zu Hause
				\item Bugfixen auf dem Rechner als Team
				\item Nachrichtenart vom rviz Plugin entnommen (Pose{\_}Stamped)
				\item Erste Version zeigen (Bild) und beschreiben
				\item Erste Version der Datenstruktur beschreiben
				\item Fehler in der UI-Entwicklung beschreiben und Verbesserungen sagen
				\item Error Handling bei schlechter Config Datei
				\item Umstellung auf existierende Kartendaten mit anderer Struktur (Karte entspricht nicht der Roboterkarte) (Robert)
				\item Verbesserung der UI mit den neuen Punkten (Zweite Version zeigen)
			\end{itemize}
			
			\subsubsection{Taster}
			Dennis\\
			\begin{itemize}
				\item Besprechung verschiedener Taster
				\item Entscheidung für Arduino
				\item Schreiben des Codes und Bauen des Tasters
				\item Einbauen des Tasters in das Weltmodell (Philipp)
				\item Verbesserung des Tasters mit Korb aus Haribo
			\end{itemize}
	
	\subsection{Ausblick}
	Eric \\
	\begin{itemize}
		\item Auftrag ist nicht ganz fertig geworden
		\item Schreiben eines Behaviours, das den Taster verwendet
		\item Verwenden von anderen Nachrichtentypen zum Senden
		\item Anpassen der Kartendaten mit Roboter-Infos
		\item Roboter kann per Text to Speech den gesuchten Gegenstand sagen
		\item Verwendung von RFID zur Erkennung der Gegenstände
	\end{itemize}
	\newpage
	\section{Teamarbeit}
	
	\subsection{Teamrollen}
		Selbsteinschätzung:\\\\
		Dennis:\\
		Basadur: \\
		Belbin: \\\\
		Robert:\\
		Basadur: \\
		Belbin: \\\\
		Philipp:\\
		Basadur: \\
		Belbin: \\\\
		Eric:\\
		Basadur: Conceptualizer, Optimizer\\
		Belbin: Weichensteller, Koordinator, Erfinder, (Spezialist)\\
	\subsection{Teamphasen}
		Dennis\\
		Grafik der Phasen einbauen.
	\subsubsection{Forming}
		Bis wohin wurde nix geschafft? Teamfindung.
	\subsubsection{Storming}
		Aushilfe vom Betreuer. Sachen kompilieren und funktionieren.
	\subsubsection{Norming}
		Kickoff-Meeting. 
	\subsubsection{Performing}
		Abschluss. Dokumentation.
	\subsection{Probleme und Lösungen}
		Philipp\\
		Komplikationen mit dem Turtlebot
		\begin{itemize}
			\item Falsche Branches (Mit Betreuer gelöst)
			\item Akku kaputt (Tausch des Akkus durch Betreuer)
			\item Kommunikation nicht möglich (Deaktivieren eines Netzwerks)
			\item Kartendaten sind nicht akkurat auf den Turtlebot zugeschnitten
		\end{itemize}
		Am Protokol orientieren.
	\section{Fazit}
		(Gemeinsam, jeder ein Absatz?)
		Wie hat die Arbeit im Team funktioniert? Anwendung der Workshop-Sachen auf reale Teamarbeit.
\end{document}